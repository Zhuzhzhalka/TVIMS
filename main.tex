%&pdfLaTeX
% !TEX encoding = UTF-8 Unicode
\documentclass[a4paper, 12pt]{article}
\usepackage[a4paper, left=5mm, right=5mm, top=5mm, bottom=5mm]{geometry}
\usepackage[T2A]{fontenc}
\usepackage[utf8x]{inputenc}
\usepackage[english, russian]{babel}
\usepackage{amsmath}
\usepackage{amssymb}
\usepackage{fancyvrb}
	
\usepackage{amsfonts} 
% or 

\usepackage{enumitem}
\usepackage{minted}
\renewcommand{\bfdefault}{b}
\usepackage{graphicx}

\usepackage{hyperref}
\hypersetup{
    colorlinks,
    citecolor=black,
    filecolor=black,
    linkcolor=black,
    urlcolor=black
}
\begin{document}
\begin{enumerate}
    \item Сигма алгебра\\
    Множество $\mathbb{F}$, элементами которого являются подмножества множества $\Omega$ (не обязательно все) наз-ся $\sigma$-алгеброй (событий), если выполнены следующие условия:
    \begin{itemize}
        \item $\Omega\in \mathbb{F}$ ($\sigma$-алгебра событий содержит достоверное событие)
        \item если $A\in \mathbb{F}$, то $\overline{A}\in \mathbb{F}$ (вместе с любым событием $\sigma$-алгебра содержит противоположное событие)
        \item если $A_1,A_2,... \in\mathbb{F}$, то $A_1\cup A_2\cup ... \in \mathbb{F}$ (вместе с любым счетным набором событий $\sigma$-алгебра содержит их объединение)
    \end{itemize}
    
    \item Вероятность\\
    Пусть $\Omega$ - пр-во элементарных исходов, $F-\sigma$-алгебра его подмножеств (событий). Вероятностью на $(\Omega,\mathbb{F})$ называется функция $P:\mathbb{F}\rightarrow\mathbb{R}$, обладающая св-вами:
    \begin{itemize}
        \item $P(A)\geq 0$ для любого события $A\in\mathbb{F}$
        \item для любого счетного набора попарно несовместных событий $A_1,A_2,A_3,...\in\mathbb{F}$ имеет место равенство\\
        \[ P(\bigcup_{i=1}^{\infty} A_i) = \sum\limits_{i=1}^{\infty} P(A_i) \]
        \item Вероятность достоверного события равна единице: $P(\Omega)=1$
    \end{itemize}
    
    \item Вероятностное пр-во\\
    Вероятностное пр-во - тройка $\langle \Omega,\mathbb{F},P\rangle$, в которой $\Omega$ - пр-во элементарных исходов, $\mathbb{F}$ - $\sigma$-алгебра его подмножеств и $P$ - вероятностная мера на $\mathbb{F}$
    
    \item Условная вероятность\\
    Условной вероятностью события $A$ при условии, что произошло событие $B$, наз-ся число
    \[ P(A|B)=\frac{P(A\cap B)}{P(B)}  \]
    Условная вероятность определена только в том случае, когда $P(B)> 0$
    
    \item Формула Байеса\\
    Пусть $H_1,H_2,...$ - полная группа событий, и $A$ - некоторое событие, вероятность которого положительна. Тогда условная вероятность того, что имело место событие $H_k$, если в результате эксперимента наблюдалось событие $A$, может быть вычислена по ф-ле
    \[ P(H_k|A)=\frac{P(H_k)P(A|H_k)}{\sum\limits_{i=1}^{\infty} P(H_i)P(A|H_i)} \]
    
    \item Случайная величина\\
    Функция $\xi:\Omega\rightarrow\mathbb{R}$ наз-ся случайной величиной, если для любого борелевского множества $B\in\mathbb{B}(\mathbb{R})$ множество $\xi^{-1}(B)$ является событием, т. е. принадлежит $\sigma$-алгебре $\mathbb{F}$
    
    \item Функция вероятности. Св-ва функции вероятности\\
    Функцией распределения случайной величины $\xi$ наз-ся функция $F_\xi:\mathbb{R}\rightarrow[0,1]$, при каждом $x\in\mathbb{R}$ равная вероятности случайной величине $\xi$ принимать значения, меньшие $x$
    \[F_\xi(x)=P(\xi<x)=P\{w:\xi(w)<x\} \]
    Свойства: \begin{itemize}
        \item Она не убывает: если $x_1<x_2$, то $F_\xi(x_1)\leq F_\xi(x_2)$
        \item Существут пределы $\lim\limits_{x\rightarrow -\infty} F_\xi(x)=0$ и $\lim\limits_{x\rightarrow +\infty} F_\xi(x)=1$
        \item Она в любой точке непрерывна слева: $F_\xi(x_0-0)=\lim\limits_{x\rightarrow x_0-0} F_\xi(x)=F_\xi(x_0)$
    \end{itemize}
    
    \item Дискретная случайная величина\\
    Случайная величина $\xi$ является дискретной (имеет дискретное распределение), если существует конечный или счетный набор чисел $a_1,a_2,...$ такой, что для всех $i$
    \[ P(\xi=a_i)>0\]\, \[ \sum\limits_{i=1}^{\infty} P(\xi=a_i)=1\]
    
    \item Попарная независимость случайных величин\\
    Случайные величины $\xi_1,...,\xi_n$ наз-ся попарно независимыми, если независимы любые две из них
    
    \item Независимость в совокупности для случайных величин\\
    Случайные величины $\xi_1,...,\xi_n$ называют независимыми (в совокупности), если для любого набора борелевских множеств $B_1,...,B_n\in \mathbb{B}(\mathbb{R})$ имеет место равенство
    \[ P(\xi_1 \in B_1,...,\xi_n \in B_n)=P(\xi_1\in B_1)\cdot...\cdot P(\xi_n\in B_n)\]
    
    \item Абсолютно непрерывная случайная величина\\
    Случайная величина $\xi$ является абсолютно непрерывной (имеет абсолютно непрерывное распределение), если существует неотрицательная функция $f_{\xi}(x)$ такая, что для любого борелевского множества $B$ имеет место равенство
    \[ P(\xi\in B)=\int\limits_B f_{\xi}(x)dx\]
    
    \item Математическое ожидание в общем случае\\
    Пусть задано вероятностное пр-во $\langle \Omega, \mathbb{F}, P\rangle$ и $\xi:\Omega\rightarrow\mathbb{R}$ - заданная на нем случайная величина. Если существует интеграл Лебега от $\xi$ по пр-ву $\Omega$, то он наз-ся математическим ожиданием
    \[ E\xi=\int\limits_\Omega \xi(w)\cdot P(d\omega)\]
    
    \item Математическое ожидание для дискретной случайной величины\\
    Математическим ожидание $E\xi$ случайной величины $\xi$ с дискретным распределением наз-ся число
    \[ E\xi=\sum\limits_k a_kp_k=\sum\limits_k a_kP(\xi=a_k),\]
    если данный ряд абсолютно сходится, т. е. если $\sum |a_i|p_i <\infty$. Иначе говорят, что математическое ожидание не существует.
    
    \item Математическое ожидание абсолютно непрерывной случайной величины\\
    Математическим ожидание $E\xi$ случайной величины $\xi$ с абсолютно непрерывным распределением наз-ся число
    \[ E\xi=\int\limits_{-\infty}^{\infty}xf_{\xi}(x)dx\]
    если этот интеграл абсолютно сходится, т. е. если $\int\limits_{-\infty}^{\infty}xf_{\xi}(x)dx<\infty$
    
    \item Дисперсия\\
    Пусть $E|\xi|^k<\infty$. Число $E\xi^k$ наз-ся моментом порядка $k$ или $k$-м моментом случайной величины $\xi$, число $E|\xi|^k$ наз-ся абсолютным $k$-м моментом, $E(\xi-E\xi)^k$ наз-ся центральным $k$-м моментом, и $E|\xi-E\xi|^k$ - абсолютным центральным $k$-м моментом случайной величины $\xi$.\\ Число $D\xi=E(\xi-E\xi)^2$ (центральный момент второго порядка) наз-ся дисперсией случайной величины $\xi$
    
    \item Ковариация\\
    Ковариацией $cov(\xi, \eta)$ случайных величин $\xi$ и $\eta$ наз-ся число $cov(\xi,\eta)=E((\xi-E\xi)(\eta-E\eta))=E\xi\eta-E\xi E\eta$
    
    \item Коэффициент корреляции\\
    Коэффициентом корреляции $\rho(\xi,\eta)$ случайных величин $\xi$ и $\eta$, дисперсии которых существуют и отличны от нуля, наз-ся число
    \[ \rho(\xi,\eta)=\frac{cov(\xi,\eta)}{\sqrt{D\xi}\sqrt{D\eta}}\]
    
    \item Квантиль. Медиана\\
    Пусть задано вероятностное пр-во $\langle\Omega, \mathbb{F}, P\rangle$ с заданным распределением ${P}^\xi$ случайной величины $\xi$. Пусть фиксировано $\alpha\in(0,1)$. Тогда квантилем уровня $\alpha$ распределения ${P}^\xi$ наз-ся число $x_\alpha \in \mathbb{R}$, такое что
    \begin{equation*}
        \begin{cases}
            P(\xi\leq x_\alpha)\geq\alpha,\\
            P(\xi\geq x_\alpha)\geq1-\alpha
        \end{cases}
    \end{equation*}
    Медианой распределения случайной величины $\xi$ наз-ся любое из чисел $\mu$ таких, что
    \begin{equation*}
        \begin{cases}
            P(\xi\leq\mu)\geq\frac{1}{2},\\
            P(\xi\geq\mu)\geq\frac{1}{2}
        \end{cases}
    \end{equation*}
    
    \item Биномиальная случайная величина\\
    $\xi\sim Bin(n,p)$\\
    Распределение вероятностей $P(\xi=k)=C_n^kp^k(1-p)^{n-k}$, $n\in\mathbb{N},\ p\in(0,1),\ k=0,1,...n$\\
    Мат. ожидание $E\xi=np$\\
    Дисперсия $D\xi=np(1-p)$
    \item Геометрическая случайная величина\\
    $\xi\sim Geom(p)$\\
    Распределение вероятностей $P(\xi=k)=p(1-p)^{k-1}$, $p\in (0,1),\ k=1,2,3...$\\
    Мат. ожидание $E\xi=\frac{1}{p}$\\
    Дисперсия $D\xi=\frac{1-p}{p^2}$
    \item Пуассоновская случайная величина\\
    $\xi\sim Pois(\lambda)$\\
    Распределение вероятностей $P(\xi=k)=e^{-\lambda}\frac{\lambda^k}{k!}$, $\lambda>0,\ k=0,1,2,...$\\
    Мат. ожидание $E\xi=\lambda$\\
    Дисперсия $D\xi=\lambda$
    \item Равномерная случайная величина\\
    $\xi\sim U(a,b)$ (равномерное непрерывное распределение)\\
    Плотность
    \begin{equation*}
        f_\xi(x)=
        \begin{cases}
            \frac{1}{b-a},\ x\in[a,b],\\
            0,\ x\not\in[a,b]
        \end{cases}
    \end{equation*}
    Мат. ожидание $E\xi=\frac{a+b}{2}$\\
    Дисперсия $D\xi=\frac{(b-a)^2}{12}$
    \item Показательная случайная величина\\
    $\xi\sim Exp(\alpha)$\\
    Плотность
    \begin{equation*}
        f_\xi(x)=
        \begin{cases}
            \alpha e^{-\alpha x},\ x\geq0,\\
            0,\ x<0
        \end{cases}
    \end{equation*}
    Мат. ожидание $E\xi=\frac{1}{\alpha}$\\
    Дисперсия $D\xi=\frac{1}{\alpha^2}$
    \item Гамма случайная величина\\
    $\xi\sim \Gamma(\lambda, \alpha)$\\
    Плотность ($\alpha>0, \lambda>0$)
    \begin{equation*}
        f_\xi(x)=
        \begin{cases}
            c\cdot x^{\lambda-1}e^{-\alpha x},\\
            0,\ x\leq0
        \end{cases}
    \end{equation*}
    $c=\frac{\alpha^\lambda}{\Gamma(\lambda)}$, где $\Gamma(\lambda)=(\lambda-1)\Gamma(\lambda-1)$ - гамма-функция Эйлера, $\Gamma(1)=1$\\
    Мат. ожидание $E\xi=\lambda\alpha$\\
    Дисперсия $D\xi=\lambda\alpha^2$
    \item Нормальная случайная величина\\
    $\xi\sim N(\alpha, \sigma^2)$\\
    Плотность $f_\xi(x)=\frac{1}{\sigma\sqrt{2\pi}} e^{-\frac{(x-\alpha)^2}{2\sigma^2}},\ x \in\mathbb{R}$\\
    Мат. ожидание $E\xi=\alpha$\\
    Дисперсия $D\xi=\sigma^2$
    
    \item Производящая функция для дискретной неотрицательной случайной величины\\
    Если $\xi$ является дискретной случайной величиной, принимающей неотрицательные целочисленные значения $\{0,1,...\}$, то производящая функция вероятностей от случайной величины $\xi$ определяется как
    \[ \psi_\xi(z)=Ez^\xi=\sum\limits_{x=0}^{\infty} p(x)z^x\]
    где $p$ - функция вероятности $\xi$. Указанный степенной ряд сходится, по крайней мере, для всех комплексных чисел $z$, т. ч. $|z|\leq 1$, иначе не обязательно сходится.
    \item Характеристическая функция случайной величины\\
    Функция $\psi_\xi(t)=Ee^{it\xi}$ вещественной переменной $t$ наз-ся характеристической функцией случайной величины $\xi$
    
    \item Теорема Бохнера-Хинчина.\\
    Пусть $\phi$ - непрерывная функция и $\phi(0)=1$. Для того, чтобы $\phi$ была характеристической функцией некоторого случайного вектора, необходимо и достаточно, чтобы $\phi$ была неотрицательно определенной, то есть при каждом целом $n>0$ для любых вещественных чисел $x_1,x_2,...,x_n$ и любых комплексных чисел $z_1,z_2,...,z_n$ выполняется неравенство $\sum\limits_{i,j=1}^n \phi(x_i-x_j)z_i\bar{z_j}\geq 0$
    
    \item Формула свёртки двух независимых случайных величин\\
    Если случайные величины $\xi_1$ и $\xi_2$ независимы и имеют абсолютно непрерывные распределения с плотностями $f_{\xi_1}(u)$ и $f_{\xi_2}(v)$, то плотность распределения суммы $\xi_1+\xi_2$ существует и равна "свертке"\ плотностей $f_{\xi_1}$ и $f_{\xi_2}$:
    \[ f_{\xi_1+\xi_2}(t)=\int\limits_{-\infty}^{\infty}f_{\xi_1}(u)f_{\xi_2}(t-u)du=\int\limits_{-\infty}^{\infty}f_{\xi_2}(u)f_{\xi_1}(t-u)du\]
    
    \item Свойство характеристических функций для суммы независимых случайных величин\\
    Характеристическая функция суммы независимых случайных величин равна произведению характеристических функций слагаемых: если случайные величины $\xi$ и $\eta$ независимы, то по свойству матожиданий ($\xi$ и $\eta$ независимы $\Rightarrow E(\xi\eta)=E\xi E\eta$),
    \[ \phi_{\xi+\eta}(t)=Ee^{it(\xi+\eta)}=Ee^{it\xi}Ee^{it\eta}=\phi_\xi(t)\phi_\eta(t)\]
    
    \item Сходимость случайных величин почти наверно\\
    Последовательность ${\xi_n}$ сходится почти наверное к случайной величине $\xi$ при $n\rightarrow\infty$ ($\xi_n\rightarrow\xi$ п.н.), если $P\{\omega|\xi_n(\omega)\rightarrow\xi(\omega)$ при $n\rightarrow\infty\}=1$
    
    \item Сходимость случайных величин по вероятности\\
    Последовательность ${\xi_n}$ сходится по вероятности к случайной величине $\xi$ при $n\rightarrow\infty$, если для любого $\varepsilon>0$: $P(|\xi_n-\xi|\geq\varepsilon)\rightarrow0$ при $n\rightarrow\infty$
    
    \item Сходимость случайных величин в среднем порядка $k$\\
    Последовательность ${\xi_n}$ сходится в среднем к случайной величине $\xi$ в $L^k$ ($k>0$), если $E|\xi_n-\xi|^k\rightarrow0$ при $n\rightarrow\infty$
    
    \item Сходимость случайных величин по распределению\\
    Последовательность ${\xi_n}$ сходится по распределению к случайной величине $\xi$, если для любого $x$ такого, что функция распределения $F_\xi$ непрерывна в точке $x$, имеет место сходимость $F_{\xi_n}(x)\rightarrow F_\xi(x)$ при $n\rightarrow\infty$ 
    
    \item Слабая сходимость случайных величин\\
    Последовательность ${\xi_n}$ сходится слабо к случайной величине $\xi$, если для любой функции $f(x)$, такой что $f(x)$ непрерывна, выполнено: $Ef(\xi_n)\rightarrow E(f_\xi)$ при $n\rightarrow\infty$
    
    \item Неравенство Маркова\\
    Если $E|\xi|<\infty$, то для любого $x>0$:
    \[ P(|\xi|\geq x)\leq \frac{E|\xi|}{x}\]
    
    \item Неравенство Чебышева\\
    Если $D\xi$ существует, то для любого $x>0$:
    \[ P(|\xi-E\xi|\geq x) \leq \frac{D\xi}{x^2}\]
    
    \item Предельная теорема Пуассона для биномиальной случайной величины\\ 
    Пусть $n\rightarrow\infty$ и $p_n\rightarrow 0$ так, что $np_n\rightarrow\lambda>0$. Тогда для любого $k\geq 0$ вероятность получить $k$ успехов в $n$ испытаниях Бернулли с вероятностью успеха $p_n$ стремится к величине $e^{-\lambda}\frac{\lambda^k}{k!}$:
    \[P(v_n=k)=C_n^kp_n^k(1-p_n)^{n-k}\rightarrow e^{-\lambda}\frac{\lambda^k}{k!}\]
    
    \item Закон больших чисел в форме Чебышева\\
    Для любой последовательности $\xi_1,\xi_2,...$ попарно независимых и одинаково распределенных случайных величин с конечным вторым моментом $E\xi_1^2<\infty$ имеет место сходимость
    \[\frac{\xi_1+...+\xi_n}{n}\rightarrow E\xi_1\]
    
    \item Центральная предельная теорема для независимых одинаково распределенных случайных величин\\
    Пусть $\xi_1,\xi_2,...$ - независимые и одинаково распределенные случайные величины с конечной и ненулевой дисперсией: $0<D\xi_1<\infty$. Тогда имеет место слабая сходимость
    \[ \frac{\xi_1+...+\xi_n-nE\xi_1}{\sqrt{nD\xi_1}}\Rightarrow N(0,1)\]
    последовательности центрированных и нормированных сумм случайных величин к стандартному нормальному распределению. 
\end{enumerate}
\end{document}
